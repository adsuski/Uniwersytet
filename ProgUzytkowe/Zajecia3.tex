\documentclass[a4paper,12pt]{article}
\usepackage[MeX]{amsmath, polski}
\usepackage[utf8]{inputenc}

%opening
\title{Programy użytkowe ćwiczenia 2}
\author{Michał Samsel}
\date{24 października 2017}

\begin{document}

\maketitle

\begin{abstract}

\end{abstract}

\section{Formuły matematyczne w TeXu}
Pojedyncze znaki dolaru:
Ułamek w tekście $ \frac{1}{x} $ \\
Oto równanie $c^{2}=a^{2}+b^{2}$



Podwojny znaki dolaru:
Ułamek $$ \frac{1}{x} $$ \\
Oto równanie $$c^{2}=a^{2}+b^{2}$$



Struktura equation:
Ułamek
\begin{equation}
\frac{1}{x}
\label{eq:rownanie1}
\end{equation}
Oto równanie
\begin{equation}
c^{2}=a^{2}+b^{2}
\label{eq:rownanie2}
\end{equation}



\begin{align}
\label{eq:partialLW}
\frac{\partial \mathcal L (w, b, \xi, \alpha, \beta)}{\partial, w} = 0 & \Rightarrow w 
		- \sum_{i=1}^n\alpha_i y_i x_i=0, \\
\label{eq:PartialLXi}
\frac{\partial \mathcal L (w, b, \xi, \alpha, \beta)}{\partial, \xi_i} =0 & \Rightarrow
		C - \alpha_i-\beta_i=0, \\
\label{eq:partialLB}
\frac{\partial \mathcal L (w, b, \xi, \alpha, \beta)}{\partial, b} = 0 & \Rightarrow \
		sum_{i=1}^n\alpha_i y_i =0.
\end{align}



Indeks górny $$x^{y} \ e^{x} \ 2^{e} \ A^{2 \times 2}$$\\
Indeks dolny $$ x_y \ a_{ij} x_{i}$$
Oba indeksy $$ x_i^2 \ x_{i^2}^{k_j} \ a_{ij}^k $$



Pierwiatek, ułamek
$$ \sqrt{ \frac{2^n}{2_n}} \neq \sqrt[ \frac{1}{3}]{1+n} $$



\section{Polecenia 2}

$$ \lim_{n \to \infty}
		\sum_{k=1}^n
		 \frac{1}{k^2}=
			\frac{\pi^2}{6} $$
			
$$ \Pi_{i=2}^{n=i^2}=
		\frac{\lim^{n \to 4}(1+\frac{1}{n})^2}{\sum k(\frac {1}{n})} $$

$$ \int_{2}^\infty 
		\frac{1}{\log_{2}x}dx=
		 \frac{1}{x}\sin x=1 - \cos^2(x) $$
		
$$ \begin{equation}
		\mathbf{}
		 \left[ \begin{array}{cccc}
		a_{11} & a_{12} & \ldots & a_{1k} \\
		a_{21} & a_{22} & \ldots & a_{2k} \\
		\vdots & \vdots & \ddots & \vdots \\
		a_{k1} & a_{k2} & \ldots & a_{kk} \\
		\end{array} \right]
	 \end{equation}
	 \begin{equation}
		\mathbf{}
		 \left[ \begin{array}{c}
		x_{1}  \\
		x_{2}  \\
		\vdots \\
		x_{k}  \\
	 \end{array} \right]
$$
\end{document}
