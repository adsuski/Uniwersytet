\documentclass[a4paper,12pt]{article}\upshape
%%\usepackage[MeX]{polski}
%%\usepackage[cp1250]{inputenc}
\usepackage{polski}
\usepackage[utf8]{inputenc}
\usepackage{graphicx}
\usepackage{multirow}
\usepackage[table]{xcolor}
\usepackage{booktabs}
\usepackage{sidecap}
\usepackage{wrapfig}
\usepackage{caption}
\usepackage{subcaption}
\usepackage{subfig}

%opening
\title{Tour de France 1919}
\author{Michał Samsel}
\date{31 października 2017}

\begin{document}

\maketitle
\begin{abstract}
\tableofcontents
\end{abstract}
	
	
	
\section{Opis}
13. Tour de France rozpoczął się 29 czerwca, a zakończył 27 lipca 1919 roku w Paryżu. Pierwsze miejsce zajął Firmin Lambot z Belgii. Mapa całej trasy. Patrz:(rysunek nr \ref{fig:mapa}.) Zwycięzcą został ,,Firmin Lambot''



\section{Etapy}
	\begin {tabular}{lclcl}
			\hline
			etap	&  data & trasa & dystans & Zwycięzca\\
			\hline
			1 & 29 czerwca & Paryż-Hawr & 388km & Jean Rossius\\
			2 & 1 lipca & Hawr-Cherbourg & 364km & Henri Pélissier\\
			3 & 3 lipca & Cherbourg-Brest & 405km & Francis Pélissier\\
			4 & 5 lipca & Brest-Les Sables-d’Olonne & 412km & Jean Alavoine\\
			\hline
		\end{tabular}
	\caption{Początkowe etapy 13. Tour de France}



\section{Mapa}
	\begin{figure}
		\begin{center}
	\includegraphics[scale=0.4]{mapa.png}
		\end{center}
	\caption{Mapa 13. Tour de France}
	\label{fig:mapa}
	\end{figure}


\end{document}
